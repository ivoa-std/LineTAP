\documentclass[11pt,a4paper]{ivoa}
\input tthdefs
\usepackage{float}
\usepackage{todonotes}

\lstloadlanguages{SQL,XML}
\lstset{flexiblecolumns=true,numberstyle=\small,showstringspaces=False,
  identifierstyle=\texttt}

\title{LineTAP: IVOA Relational Model for Spectral Lines}

% see ivoatexDoc for what group names to use here
\ivoagroup{DAL}

\author{Castro Neves, M.}
\author{Moreau, N.}
\author{Demleitner, M.}

\editor{Margarida Castro Neves, Nicolas Moreau}


\previousversion{This is the first public release}

\begin{document}
\begin{abstract}

This document proposes a relational schema to describe spectral line
transitions that can be queried using the TAP protocol. Its purpose is,
parting from VAMDC and SSLDM models, to present a simpler way to query
for spectral line data using VO applications. The underlying model is
rooted in the widely-depolyed VAMDC, and the intent is that at least the
atomic and molecular data from VAMDC can easily be re-published using
LineTAP.

%Among the existing services, SSLDM/SLAP has very few services deployed
%and is not being used. On the other side, VAMDC is has plenty of
%services and spectral line information, but its data model, which is
%very detailed, is also very complex.  Therefore the need of a  simpler
%model, using already established protocols like TAP for querying and
%retrieving the information. Besides the proposed set of parameters, a
%mapping from VAMDC data model to LineTAP model is also provided.

\end{abstract}


%\section*{Acknowledgments}


\section*{Conformance-related definitions}

The words ``MUST'', ``SHALL'', ``SHOULD'', ``MAY'', ``RECOMMENDED'', and
``OPTIONAL'' (in upper or lower case) used in this document are to be
interpreted as described in IETF standard RFC2119 \citep{std:RFC2119}.

The \emph{Virtual Observatory (VO)} is a
general term for a collection of federated resources that can be used
to conduct astronomical research, education, and outreach.
The \href{http://www.ivoa.net}{International
Virtual Observatory Alliance (IVOA)} is a global
collaboration of separately funded projects to develop standards and
infrastructure that enable VO applications.


\section{Introduction}

The Simple Line Access Protocol SLAP \citep{2010ivoa.specQ1209O}
currently is the VO
recommendation for querying spectral line collections. 
It is based on the Simple Spectral Line Data
Model SSLDM \citep{2010ivoa.spec.1209O}, which defines the underlying
data model.
As  in SSLDM, a \emph{spectral line} in this document is considered to
be the result of a (radiative) transition between two energy levels.

More than ten years after the protocol's definition, there are still
very few SLAP services registred in the VO.
On the other hand, the Virtual Atomic and Molecular Data Center
VAMDC\todo{reference?} offers a great amount of spectral line
data.  Making this data available to VO clients without major extra
tooling is certainly desirable.  

While the query part of VAMDC clearly betrays its origins in VO
standards and thus might readily be integrated into the VO protocol
stack, the service output comes in a very comprehensive derivative of
the XML Schema for Atomic, Molecular and Solid Data XSAMS
\citep{XSAMS:Docs}; in particular, its tree-like nature complicates
casual use.  In addition, many interesting use cases can already be
satisfied with a simple relational mapping of XSAMS. \todo{Prior art:
HTML table XSLT from VAMDC -- do we want to reference it?}

This document proposes LineTAP, a simple way to access spectral line
data through a VO service employing such a simplified relational
mapping.  The resulting table schema is presented in
section~\ref{sect:quantities}, while the mapping between our columns and the
VAMDC-XSAMS Data Model is given in section~\ref{sect:mapping}.

When accessed using the Table Access Protocol TAP
\citep{2019ivoa.spec.0927D}, the table can be queried using the
expressive SQL-derived query language ADQL, while query results are
available in the VOTable format, easily readable by VO client
applications.  Line databases accessible in this way can be registered
in the VO Registry.  The detailed rules for this registration, and
recommendations for how to discover LineTAP services, are given in
section~\ref{sect:regmatters}.



\section{Use Cases}

LineTAP really only has a single use case, the discovery of spectral
lines for identification purposes.  To structure standards development,
we discuss some situations specifically:

\subsection{Identifying a Single Line}

A user sees a feature in a spectral with known (and realiable) spectral
calibration and now wants to know what might possibly be responsible for
it.  Hence, they query a narrow spectral range and retrieve all known
lines from all services.

To select which of the candidate lines are plausible matches, users
would inspect line metadata such as the originating atom or element, the
ionisation state, and perhaps oscillator strengths.

\subsection{Getting Properties of Well-Known Lines}

A user wants to display, say, the Lyman series over a plot of a
spectrum.  Hence, a client needs to discover which service holds such
data, select the appropriate records -- presumably by their properties,
perhaps even by their name --, and retrieve them.  If multiple services
hold the desired data, it might need to reconcile differing
specifications.


\subsection{Retrieving Spectral Lines for Cross-Identification}

Users may have various reasons to retrieve a larger number of spectral
lines:

\begin{itemize}
\item When analysing a given spectrum, selecting spectral lines that may
fit the ones in the spectrum, for instance to establish the source's
chemistry or physical state.  Depending on the prior knowledge of the
source, they will want to constrain the matches to specific species in
specific ionisation (or even excitation) states.

\item When estimating the redshift of an object, features found in the
spectrum need to be matched to the rest wavelengths.

\item When computing theoretical spectra, a comparison to the (observed)
ground truth is desirable.
\end{itemize}

The challenge in all these cases is that, say, displaying all lines
known obviously is impossible and would not help users in any way.
Hence, the client needs to have some idea of which lines can be expected
to be strong given the physics of the emission's source region.

Selecting the lines before retrieval is a significant optimisation in
this case, as in wider spectra at least hundreds of thousands of lines
will be within the spectral range, while it probably rarely makes sense
to plot more than a hundred or so.  Hence, careful selection of lines
can reduce the volume of data transferred and processed by the client by
several orders of magnitude.

To make good on this promise, the tables need to be queriable such that
lines suspected to be strong for some combination of chemistry,
temperature, and pressure can be filtered out with some accuracy.


\subsection{Finding spectral lines for specific species}

Using a mass spectrometer, researchers find a molecule with the
sum formula C$_{16}$H$_{10}$  in a comet particle.  They now want to
figure out whether any line in the spectrum of the coma of the parent
object corresponds to some molecule with that sum formula.

A specific challenge in this use case is the characterisation of the
molecule.  While InChi labels and keys help to some degree, user
acceptance of them at this point is at least questionable; also,
searching by sum formula is impossible using InChis\todo{A claim
made without actually knowing InChi)}.  

As a fallback mechanism, and perhaps also as a means of characterising
servers, users may want to discover all species served by a specific
server.

\subsection{Credit}

In particular to provide an incentive to contribute to the global
repository of line data, it should be as simple as possible for users to
give credit to the contributors of line data.


\subsection{Non-Use Cases}

This specification differs from VAMDC in that it does not attempt to
cover all possible uses of spectral line data.  In particular, no
attempt is made to

\begin{itemize}
\item Publish sufficient information to feed sophisticated,
high-precision atmosphere models.
\item Deal with solid-state spectroscopy.
\item Publish spectra of non-electromagnetic messengers.
\end{itemize}


\section{Spectral Line Data}\label{sect:quantities}

In the following table the selected quantities based on IVOA use cases
are defined, including units, data types and mandatory status:

\todo{ method: define other method options besides experiment and theory?}

\begin{table}[H]
\small
%\begin{center}
\begin{tabular}{|l|l|l|p{0.8cm}|p{4cm}|}%{l l}% p{0.4\linewidth}}
\hline
\textbf{column name} & \textbf{unit} & \textbf{type} & \textbf{man da tory} & 
\textbf{description} \\
\hline
\hline
\texttt{title} & & string & & name of the species originating  the line.  \\
\hline
\texttt{vacuum\_wavelength} & Angstrom & double & yes &  wavelength in vacuum \\
\hline
\texttt{vacuum\_wavelength\_error} & & double & & integrated error for the spectral location \\
\hline
\texttt{method} && string& & method the wavelength was obtained with:
\textit{experiment} (observed), or \textit{theory} (calculated from theoretical models) \\ 
\hline
\texttt{stoichiometric\_formula} & & string & &  the symbol of the chemical element or the 
molecule formula\\
\hline
\texttt{ion\_charge} & & integer & yes & ionisation level\\
\hline
\texttt{mass\_number} & & integer &  & atomic or molecular mass number\\
\hline
\texttt{upper\_state\_configuration} & & string & & upper state configuration\\
\hline
\texttt{lower\_state\_configuration} & & string & & lower state configuration\\
\hline
\texttt{upper\_energy} & J  & double & & energy of the upper state \\
\hline
\texttt{lower\_energy} & J & double & & energy of the lower state\\
\hline
\texttt{inchi} & & string & yes & chemical species inchi \\
\hline
\texttt{inchikey} & & string & yes & chemical species inchikey \\
\hline
\texttt{einstein\_a} & & double  & &  Einstein A coefficient\\
\hline
\texttt{oscillator\_strength} & & double & & oscillator strength of radiative transition \\
\hline
\texttt{weighted\_oscillator\_strength} & & double  & &  Weighted oscillator strength of 
radiative transition \\
\hline
\texttt{line\_strength} & & string & & Total absorption by a spectra line \\
\hline
\texttt{line\_reference\_doi} & & string & & Digital Object Identifier of bibliography source \\
\hline
\texttt{line\_reference\_uri} & & string & & Web link to the publication\\
\hline
\end{tabular}

%\end{center}
\end{table}

\begin{itemize}
\item \texttt{vacuum\_wavelength} 
Wavelength value is always in vacuum, and  is stored in the database in
Angstrom.  An  ADQL user-defined function is provided to convert this
wavelengthto the desired unit. They will be described in
\ref{sec:Protocol}.

\item \texttt{vacuum\_wavelength\_error} is the integrated error for the wavelength.
\todo {comment on symmetry/assymmetry of errors} 

\item \texttt{stoichiometric\_formula} is the chemical element symbol
for atoms. For molecules it is an alphabetical suite of the atomic
constituents followed by the total number of their occurences (e.g.
HCOOH => CH2O2 ).

\item \texttt{ion\_charge} is the ionization level of the species

\item \texttt{inchi} is the The IUPAC International Chemical Identifier
(InChI), which  provides unique labels for well-defined chemical
substances (\cite{INCHI}). It is meant to be human readable, but
depending of the molecule it can be very long.

\item \texttt{inchiKey} is a character signature based on a hash code of
the InChI string. It will be used to uniquely identify species.

\item \texttt{mass\_number} The mass number of an atom or the sum of
each atomic mass numbers of the elements forming a molecule.

\item \texttt{upper\_state\_configuration} a text string describing the
quantum state of the state with higher energy (e.g. 
\item \texttt{lower\_state\_configuration} 
\item \texttt{upper\_energy} 
\item \texttt{lower\_energy} 

\item \texttt{einstein\_a}
\item \texttt{oscillator\_strength}
\item \texttt{weighted\_oscillator\_strength}
\item \texttt{line\_strength}

\end{itemize}


\section{Protocol }
\label{sec:Protocol}
\subsection{ Queries: LineTAP }

\subsection{User-defined functions}

\begin{itemize}
\item ivo\_spectral(value, unit)\\
Converts energy values in J to values in other units. Example:\\
\begin{quote}
SELECT ivo\_spectral(energy, 'Angstrom') AS lambda \\
 WHERE ivo\_spectral(energy, 'MHz') BETWEEN 88 AND 108
\end{quote}

\end{itemize}

\todo{ description, examples, translation VAMDC Queries}



\section{LineTAP use case examples}


\section{Mapping from VAMDCXSAMS}
\label{sect:mapping}

The quantities used in the mapping belong to the following branches of VAMDCXSAMS:\\
\\
\textit{XSAMSData.Species.Atoms.Atom}  referred as ATOM,\\
\textit{XSAMSData.Species.Atoms.AtomicState}  referred as ATOMICSTATE,\\
\textit{XSAMSData.Species.Molecules.Molecule}, referred as MOLECULE,\\
\textit{XSAMSData.Processes.Radiative.RadiativeTransition}, referred as RADTRANS.\\
\textit{XSAMSData.Sources}, referred as RADTRANS.\\


The VAMDC-TAP query language defines a list of keywords, called
\href{https://standards.vamdc.eu/dictionary/returnables.html}{Returnables},
listing the quantities that can be returned by a service.They correspond
to an element in the XSAMS data model.  The returnables supported by a
service are declared by a service manager when he deploys the software
on its database, and saved in the VAMDC registry. \\

Listed below are the mappings for the atomic quantities defined in
\ref{quantities} (at the moment only atoms, not molecules):

\renewcommand{\descriptionlabel}[1]{\hspace{\labelsep}\texttt{#1}}
\begin{description}

\item [vacuum\_wavelength] (in J)\hfill\\
    \textit{data model:} RADTRANS.EnergyWavelength\\
	\textit{returnables needed:} RadTransWavelength, RadTransWavelength.Vacuum, 
RadTransWavelengthAirToVac, RadTransWavelengthUnit\\
	\textit{constraints:} RadTransWavelengthVacuum = true; else use 
RadTransWavelengthAirToVac. \\
	
\item [vacuuml\_wavelength\_error]  \hfill\\\todo{Wavelength error? How to get it from VAMDC data?}

    \textit{data model:} RADTRANS.EnergyWavelength\\
	\textit{returnables needed:} RadTransWavelength, RadTransWavelength.Vacuum, 
RadTransWavelengthAirToVac, RadTransWavelengthUnit\\
	\textit{constraints:} RadTransWavelengthVacuum = true; else use 
RadTransWavelengthAirToVac.\\
	
\item [method] \hfill\\
	\textit{data model:} Methods.Method.Category\\
	\textit{returnables needed:} RadTransWavelengthMethod\\
	\textit{constraints:} 
Methods.Method.MethodID=RadTransWavelength.Wavelength@methodRef\\

\item [inchi] \hfill\\
	\textit{data model:} ATOM.Isotope.Ion.InChi,  
MOLECULE.MolecularChemicalSpecies.InChI\\
        \textit{returnables needed:} AtomInchi, MoleculeInChi\\
	\textit{constraints:} RadTrans.SpeciesRef = Atom.SpeciesID or MoleculeSpeciesID
	
\item [inchikey] \hfill\\
	\textit{data model:} ATOM.Isotope.Ion.InChiKey, 
MOLECULE.MolecularChemicalSpecies.InChIKey \\
         \textit{returnables needed:} AtomInchiKey, MoleculeInChiKey\\
         \textit{constraints:} RadTrans.SpeciesRef = Atom.SpeciesID or  Molecule.SpeciesID

\item [stoichiometric\_formula] \hfill\\
	\textit{data model:} ATOM.ChemicalElement.ElementSymbol,\\ MOLECULE. 
MolecularChemicalSpecies.StoichiometricFormula\\
	\textit{returnables needed:} RadTrans.SpeciesRef, Atom.SpeciesID, 
Molecule.SpeciesID, Atom.Symbol, Molecule.MoleculeStoichiometricFormula\\
	\textit{constraints:}  RadTrans.SpeciesRef = Atom.SpeciesID or  Molecule.SpeciesID

\item [ion\_charge]\hfill\\
	\textit{data model:} ATOM.Isotope.Ion.IonCharge or 
MOLECULE.MolecularChemicalSpecies.IonCharge \\
	\textit{returnables needed:} RadTransSpeciesRef, AtomSpeciesID, AtomIonCharge, 
MoleculeSpeciesID, MoleculeIonCharge \\
	\textit{constraints:}RadTransSpeciesRef=AtomSpeciesID or  Molecule.SpeciesID

	\item [mass\_number] (atoms)\hfill\\
	\textit{data model:} ATOM.Isotope.IsotopParameters.MassNumber \\
	\textit{returnables needed:} AtomMassNumber\\
	\textit{constraints:}RadTransSpeciesRef=AtomSpeciesID

	\item [upper\_state\_configuration]\hfill\\
	\textit{data model:} ATOMICSTATE.Description, MOLECULARSTATE.Description\\
	\textit{returnables needed:} RadTransUpperStateRef, AtomStateID, 
MoleculeStateID,\\ 
AtomStateDescription, MoleculeStateDescription\\
	\textit{constraints:}  RadTransUpperStateRef = AtomStateID

	\item [lower\_state\_configuration]\hfill\\
	\textit{data model:} ATOMICSTATE.Description, MOLECULARSTATE.Description\\
	\textit{returnables needed:} RadTransUpperStateRef, AtomStateID, 
MoleculeStateID,\\ 
AtomStateDescription, MoleculeStateDescription\\
	\textit{constraints:}  RadTransUpperStateRef = AtomStateID or MolecularStateID
	
	\item [upper\_state\_energy]\hfill\\
	\textit{data model:} ATOMICSTATE.AtomicNumericalData.StateEnergy, \\ 
MOLECULARSTATE.MolecularStateCharacterisation.StateEnergy\\
	\textit{returnables needed:} RadTransUpperStateRef, AtomStateID, 
MoleculeStateID\\
	\textit{constraints:} 
 RadTransUpperStateRef = AtomStateID or MoleculeStateID. 
 Original value in $cm^{-1}$ has to be converted to $J$
	
	\item [lower\_state\_energy]\hfill\\
	\textit{data model:} ATOMICSTATE.AtomicNumericalData.StateEnergy, \\ 
MOLECULARSTATE.MolecularStateCharacterisation.StateEnergy\\
	\textit{returnables needed:} RadTransLowerStateRef, AtomStateID,  
MoleculeStateID\\
	\textit{constraints:}  RadTransLowerStateRef = AtomStateID or MoleculeStateID.
	Original value in $cm^{-1}$ has to be converted to $J$

	\item [einstein\_a]\hfill\\
	\textit{data model:}  RADTRANS.Probability.TransitionProbabilityA\\
	\textit{returnables needed:} RadTransProbabilityA\\
	\textit{constraints:}

	\item [oscillator\_strength]\hfill\\
	\textit{data model:}  RADTRANS.Probability.TransitionOscillatorStrength\\
	\textit{returnables needed:} RadTransProbabilityOscillatorStrength\\
        \textit{constraints:}

	\item [weighted\_oscillator\_strength]\hfill\\
	\textit{data model:}  RADTRANS.Probability.TransitionWeightedOscillatorStrength\\
	\textit{returnables needed:} RadTransProbabilityWeightedOscillatorStrength\\
 	   \textit{constraints:}

	\item [line\_reference\_doi]\hfill\\
	\textit{data model:} SOURCES.Source.DigitalObjectIdentifier\\
	\textit{returnables needed:} SourceDOI\\
	\textit{constraints:}

	\item [line\_reference\_uri]\hfill\\
	\textit{data model:} SOURCES.Source.UniformResourceIdentifier\\
	\textit{returnables needed:} SourceURI\\
	\textit{constraints:}

	\item [line\_strength]\hfill\\
	\textit{data model:} RadTrans.Probability.TransitionLineStrength\\
	\textit{returnables needed:} RadTransProbabilityLineStrength\\
        \textit{constraints:}

\item [title]\hfill\\
	a string composed by the values of  \texttt{element\_symbol} and \texttt{ion\_charge}


\end{description}

\section{LineTAP and the VO Registry}
\label{sect:regmatters}

\subsection{Registering LineTAP-conforming Tables}

LineTAP tables are registered using VODataService
\citep{2010ivoa.spec.1202P} tablesets, where the table utype is set to
$$\hbox{\verb|ivo://ivoa.net/std/linetap#table-1.0|}.$$

The tableset is normally contained in a VODataService \xmlel{CatalogService}
record with a TAP capability, and this capability normally is an auxiliary
capability as per DDC \citep{2019ivoa.rept.0520D}.  For one-table
services a full TAPRegExt \citep{2012ivoa.spec.0827D} capability is also
allowed; other resource types can be used for registration as
appropriate.

Further capabilities, for instance for full VAMDC or legacy SLAP
services, may be given in the same record.

An example for a registry record in VOResource, for the case of
using an auxiliary capability referencing a main TAP service comes with
this document\footnote{\auxiliaryurl{example-record.xml}}.

The noteworthy points in the record are:

\begin{itemize}
\item A \xmlel{relationship} element referencing the main TAP service 
through which the service is queriable as per DDC:
\begin{lstlisting}[language=XML,basicstyle=\footnotesize]
<relationship>
  <relationshipType>served-by</relationshipType>
  <relatedResource ivo-id="ivo://org.gavo.dc/tap"
    >GAVO Data Center TAP service</relatedResource>
</relationship>
\end{lstlisting}

\item The declaration for the auxiliary capability, including the access
URL so clients to not need to follow the relationship just declared if
all they need is the access URL:
\begin{lstlisting}[language=XML,basicstyle=\footnotesize]
<capability standardID="ivo://ivoa.net/std/TAP#aux">
   <interface role="std" version="1.1" xsi:type="vs:ParamHTTP">
     <accessURL use="base">http://dc.zah.uni-heidelberg.de/tap</accessURL>
   </interface>
</capability>
\end{lstlisting}

\item Most importantly, the declaration of the table utype that lets
clients discover that this particular table contains EPNCore data:
\begin{lstlisting}[language=XML,basicstyle=\footnotesize]
<table>
  <name>toss.ivoa_lines</name>
  <title>TOSS</title>
  <description> The EPN-TAP 2.0 version of...</description>
  <utype>ivo://ivoa.net/std/linetap#table-1.0</utype>
  ...
</table>
\end{lstlisting}
\end{itemize}

That in the example record, the resource description is identical to the
description of the schema, which again is identical to the description
of the table is an artefact of LineTAP registrations being single-table
and is thus to be expected in most registrations of this type.


\subsection{Discovering LineTAP services}

LineTAP consumers in general are interested in TAP endpoints and table names for
lineTAP services.  By our registration pattern, this translates into
resources with TAP capabilities that have a standard key for version 1
LineTAP in a table utype.

Translated into RegTAP \citep{2019ivoa.spec.1011D}, the following query
would return TAP access URLs and the table names, using ADQL 2.1 CTEs
for readability:

\begin{lstlisting}[language=SQL]
SELECT DISTINCT table_name, access_url
FROM rr.res_table
  NATURAL JOIN rr.capability
  NATURAL JOIN rr.interface
WHERE
  table_utype='ivo://ivoa.net/std/linetap#table-1.%'
  AND standard_id LIKE 'ivo://ivoa.net/std/tap%'
  AND intf_role='std'
\end{lstlisting}

The \texttt{DISTINCT} in the main query is a rough filter that removes
entries duplicated because their tables are registred both in the main
TAP record and in an auxiliary capability.  

The regular expression in the utype match is to make sure minor version
increments do not prevent service discovery; by IVOA versioning rules,
all lineTAP services of minor version 1 can be operated by all lineTAP
clients of version 1.  We do not constrain the version of the TAP
service; clients may want to adapt the TAP discovery pattern to match
their specific needs.



\appendix
\section{Changes from Previous Versions}

No previous versions yet.
% these would be subsections "Changes from v. WD-..."
% Use itemize environments.


\bibliography{ivoatex/ivoabib,ivoatex/docrepo, localrefs}

\end{document}
