\documentclass[11pt,a4paper]{ivoa}
\input tthdefs
\usepackage{float}


\title{Mapping VAMDC spectral line data to SLAP}

% see ivoatexDoc for what group names to use here
\ivoagroup{???? group ????}

\author{Margarida Castro Neves, Nicolas Moreau}

\editor{Margarida Castro Neves, Nicolas Moreau}

% \previousversion[????URL????]{????Funny Label????}
\previousversion{This is the first public release}

\begin{document}
\begin{abstract}
VAMDC contains  a great amount of spectral line data, that could be made more
accessible from VO Applications with VO protocols. This document proposes a mapping
between relevant VAMDCXSAMS Data Model and a new set of selected parameters  in
order to define a spectral line protocol to access this information.


\end{abstract}


%\section*{Acknowledgments}


\section*{Conformance-related definitions}

The words ``MUST'', ``SHALL'', ``SHOULD'', ``MAY'', ``RECOMMENDED'', and
``OPTIONAL'' (in upper or lower case) used in this document are to be
interpreted as described in IETF standard RFC2119 \citep{std:RFC2119}.

The \emph{Virtual Observatory (VO)} is a
general term for a collection of federated resources that can be used
to conduct astronomical research, education, and outreach.
The \href{http://www.ivoa.net}{International
Virtual Observatory Alliance (IVOA)} is a global
collaboration of separately funded projects to develop standards and
infrastructure that enable VO applications.


\section{Introduction}

SLAP (Simple Line Access Protocol)\citep{2010ivoa.specQ1209O} is the VO standard for Spectral Line querying. SSLDM (Simple Spectral Line Access Protocol)\citep{2010ivoa.spec.1209O} is the VO data model for spectral lines.
Currently there are very few working services in the VO that use this protocols, meaning that the amount of spectral line data in the VO is small. 
On the other side, VAMDC services offer a great amount of spectral line data, and  so it's desirable to access this data in a VO-way. 
The queries to the VAMDC services might be simple, however, the data returned is much more than the necessary data for the common use cases in VO, besides the XSAMS \citep{XSAMS:Docs} output format being quite complex.
This document proposes a simple way to access VAMDC data through a VO service. The data returned is a selection of the common used spectral line metadata that will be mapped from the \href{https://standards.vamdc.eu/#data-model}{VAMDC-XSAMS Data Model} into an easy accessible format. A query protocol is used  to retrieve this data,  that will be returned in a VOTABLE format, easily readable by VO client applications.

\section{Use Cases}

In most use cases, the goal is to search for spectral lines within a wavelength range:

\begin{itemize}

\item selecting  spectral lines that fit the ones in the spectrum.
Normally when searching for lines within a wavelength range, a great quantity of lines are returned, with very similar wavelengths. It's necessary to select the one that really represents the transition that caused the line in the spectrum.
\item areas: Stars, galaxies, ISM, planets?
\item calculating redshift
\item comparing theoretical spectra to observed spectra
\item classify the star type according to spectral lines?
\item spectral lines from a specific atom (or molecule)
\item spectral lines from a specific ionisation level
\item spectral lines with specific (other) parameter
\end{itemize}

Another use case to consider is to get a list of all available chemical species 
from a service, as a first step before looking for spectral lines.

For any retrieved spectral line, the reference of the source article where it was published
should always be provided if available.

\section{Protocol, Service discovery and queries - (need better title)}
TO DO -> Markus

discovery -> VO Registry
querying ->  TAP ? (ObsCore?)
What is done in the Client, what is done in the server, how it would work


\section{Mapping}



\subsection{Proposed quantities to be retrieved}\label{quantities}

Relevant quantities are to be chosen based on IVOA use cases. We define the following quantities (based on use cases) to be mapped from VAMDCXSAMS \textit{returnables needed:}

\begin{table}[H]
%\small
\begin{center}
\begin{tabular}{l l}% p{0.4\linewidth}}
\texttt{title} & the title of the line \\
\texttt{wavelength} & wavelength in vacuum \\
\texttt{inchi} & chemical species inchi \\
\texttt{inchikey} & chemical species inchikey \\
\texttt{element\_symbol} & symbol describing the chemical element \\
\texttt{ion\_charge} & ionisation level\\
\texttt{atom\_mass\_number} & atom mass number\\
\texttt{upper\_state\_description} & upper state description\\
\texttt{lower\_state\_description} & lower state description\\
\texttt{upper\_energy} & energy of the upper state \\
\texttt{lower\_energy} & energy of the lower state\\
\texttt{einstein\_a} & Einstein A coefficient\\
\texttt{oscillator\_strength} & oscillator strength of radiative transition \\
\texttt{weighted\_oscillator\_strength} &  Weighted oscillator strength of radiative transition \\
\texttt{line\_strength} & Total absorption by a spectra line \\
\texttt{line\_reference\_doi} & Digital Object Identifier of bibliography source \\
\texttt{line\_reference\_uri} & Web link to the publication\\
\end{tabular}

\end{center}
\end{table}

Returnables corresponding to a "DataType" in the XSAMS schema (numerical values) have additional categories, besides the value. Following categories will be also used in the mapping, \textit{Keyword} being the returnable keyword:

 \begin{itemize}
\item \textit{Keyword}Unit  - the units of the value
\item \textit{Keyword}Method - the method used to obtain this value (e.g. experiment, theory, etc)
\end{itemize}


\subsection{Mapping from VAMDCXSAMS}

The quantities used in the mapping belong to the following branches of VAMDCXSAMS:\\\\
\textit{XSAMSData.Species.Atoms.Atom}  referred as ATOM,\\
\textit{XSAMSData.Species.Atoms.AtomicState}  referred as ATOMICSTATE,\\
\textit{XSAMSData.Species.Molecules.Molecule}, referred as MOLECULE,\\
\textit{XSAMSData.Processes.Radiative.RadiativeTransition}, referred as RADTRANS.\\

The VAMDC-TAP query language defines a list of keywords, called \href{https://standards.vamdc.eu/dictionary/returnables.html}{Returnables}, listing the quantities that can be returned by a service.They correspond to an element in the XSAMS data model. The returnables supported by a service are declared by a service manager when he deploys the software on its database, and saved in the VAMDC registry. \\

Listed below are the mappings for the atomic quantities defined in \ref{quantities} (at the moment only atoms, not molecules):

\renewcommand{\descriptionlabel}[1]{\hspace{\labelsep}\texttt{#1}}
\begin{description}

\item [wavelength] (in vacuum)\hfill\\
    \textit{data model:} RADTRANS.EnergyWavelength\\
	\textit{returnables needed:} RadTransWavelength, RadTransWavelength.Vacuum, RadTransWavelengthAirToVac, RadTransWavelengthUnit, RadTransWavelengthMethod\\
	\textit{constraints:} RadTransWavelengthVacuum = true; else use RadTransWavelengthAirToVac.\\

	\item [inchi] \hfill\\
	\textit{data model:} ATOM.Isotope.Ion.InChi\\
    \textit{returnables needed:} AtomInchi\\
	\textit{constraints:} \\

	\item [inchikey] \hfill\\
	\textit{data model:} ATOM.Isotope.Ion.InChiKey\\
    \textit{returnables needed:} AtomInchiKey\\
    \textit{constraints:} \\

\item [element\_symbol] \hfill\\
	\textit{data model:} ATOM.ChemicalElement.ElementSymbol\\
	\textit{returnables needed:} RadTransSpeciesRef, Atom.SpeciesID, Atom.Symbol
	\textit{constraints:}  RadTransSpeciesRef = Atom.SpeciesID

\item [ion\_charge]\hfill\\
	\textit{data model:} ATOM.Isotope.Ion.IonCharge\\
	\textit{returnables needed:} RadTransSpeciesRef, AtomSpeciesID, AtomIonCharge\\
	\textit{constraints:}

	\item [atom\_mass\_number]\hfill\\
	\textit{data model:} ATOM.Isotope.IsotopParameters.MassNumber\\
	\textit{returnables needed:} AtomMassNumber\\
	\textit{constraints:}

	\item [upper\_state\_description]\hfill\\
	\textit{data model:} ATOMICSTATE.Description\\
	\textit{returnables needed:} RadTransUpperStateRef, AtomStateID\\
	\textit{constraints:}  

\item [lower\_state\_description]\hfill\\
	\textit{data model:} ATOMICSTATE.Description\\
	\textit{returnables needed:} RadTransLowerStateRef, AtomStateID\\
	\textit{constraints:}  

\item [upper\_energy]\hfill\\
	\textit{data model:} ATOMICSTATE.AtomicNumericalData.StateEnergy\\
	\textit{returnables needed:} RadTransUpperStateRef, AtomStateID, AtomStateEnergyUnit\\
	\textit{constraints:}  RadTransUpperStateRef = AtomStateID

\item [lower\_energy]\hfill\\
	\textit{data model:} ATOMICSTATE.AtomicNumericalData.StateEnergy\\
	\textit{returnables needed:} RadTransLowerStateRef, AtomStateID, AtomStateEnergyUnit\\
	\textit{constraints:}  RadTransLowerStateRef = AtomStateID

\item [einstein\_a]\hfill\\
	\textit{data model:}  RADTRANS.Probabilitty.TransitionProbabilityA\\
	\textit{returnables needed:} RadTransProbabilityA\\
	\textit{constraints:}

\item [oscillator\_strength]\hfill\\
	\textit{data model:}  RADTRANS.Probabilitty.TransitionOscillatorStrength\\
	\textit{returnables needed:} RadTransProbabilityOscillatorStrength\\
    \textit{constraints:}

\item [weighted\_oscillator\_strength]\hfill\\
	\textit{data model:}  RADTRANS.Probabilitty.TransitionWeightedOscillatorStrength\\
	\textit{returnables needed:} RadTransProbabilityWeightedOscillatorStrength\\
    \textit{constraints:}

\item [line\_reference\_doi]\hfill\\
	\textit{data model:} SOURCES.Source.DigitalObjectIdentifier\\
	\textit{returnables needed:} SourceDOI\\
	\textit{constraints:}

\item [line\_reference\_uri]\hfill\\
	\textit{data model:} SOURCES.Source.UniformResourceIdentifier\\
	\textit{returnables needed:} SourceURI\\
	\textit{constraints:}

\item [line\_strength]\hfill\\
	\textit{data model:} RadTrans.Probabilitty.TransitionLineStrength\\
	\textit{returnables needed:} RadTransProbabilityLineStrength\\
    \textit{constraints:}

\item [title]\hfill\\
	a string composed by the values of  \texttt{element\_symbol} and \texttt{ion\_charge}


\end{description}

\appendix
\section{Changes from Previous Versions}

No previous versions yet.
% these would be subsections "Changes from v. WD-..."
% Use itemize environments.


\bibliography{ivoatex/ivoabib,ivoatex/docrepo, localrefs}

\end{document}